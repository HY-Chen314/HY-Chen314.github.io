\documentclass[12pt]{article}
\usepackage[UTF8]{ctex}
\usepackage{amsmath,amssymb,amsthm}
\usepackage{enumitem}
\usepackage{hyperref}
\usepackage{booktabs}

\setlength{\parindent}{0pt}

\title{积分竞赛典型题目精解}
\author{数学教研组}
\date{\today}

% 自定义定理环境
\newtheorem{method}{解题方法}
\newtheorem{trick}{技巧}

\begin{document}

\maketitle

\section*{说明}
本文档包含三类积分竞赛典型题目的完整解答过程,所有推导均经过验证。

\section{有理函数积分}

\begin{method}[部分分式分解]
适用于分母可因式分解的有理函数积分
\end{method}

\subsection*{题目1}
计算积分:
\[ \int \frac{2x+3}{(x-1)(x^2+4)} \, dx \]

\subsubsection*{解答}
\begin{enumerate}[label=\arabic*]
\item 设部分分式:
\[ \frac{2x+3}{(x-1)(x^2+4)} = \frac{A}{x-1} + \frac{Bx+C}{x^2+4} \]

\item 确定系数:
\begin{align*}
2x + 3 &= A(x^2 + 4) + (Bx + C)(x - 1) \\
&= (A + B)x^2 + (-B + C)x + (4A - C)
\end{align*}
得到方程组:
\[
\begin{cases}
A + B = 0 \\
-B + C = 2 \\
4A - C = 3
\end{cases}
\Rightarrow
\begin{cases}
A = 1 \\
B = -1 \\
C = 1
\end{cases}
\]

\item 逐项积分:
\[
\int \left( \frac{1}{x-1} + \frac{-x+1}{x^2+4} \right) dx = \ln|x-1| - \frac{1}{2}\ln(x^2+4) + \frac{1}{2}\arctan\left(\frac{x}{2}\right) + C
\]
\end{enumerate}

\section{三角换元积分}

\begin{trick}[万能代换]
对于包含多种三角函数的积分,可尝试$ t = \tan\frac{x}{2} $
\end{trick}

\subsection*{题目2}
计算:
\[ \int \frac{dx}{2+\cos x} \]

\subsubsection*{解答}
\begin{align*}
\text{令} \quad t &= \tan\frac{x}{2}, \quad \cos x = \frac{1-t^2}{1+t^2}, \quad dx = \frac{2}{1+t^2}dt \\
\int \frac{dx}{2+\cos x} &= \int \frac{\frac{2}{1+t^2}}{2 + \frac{1-t^2}{1+t^2}} dt \\
&= \int \frac{2}{3 + t^2} dt \\
&= \frac{2}{\sqrt{3}} \arctan\left( \frac{t}{\sqrt{3}} \right) + C \\
&= \frac{2}{\sqrt{3}} \arctan\left( \frac{\tan\frac{x}{2}}{\sqrt{3}} \right) + C
\end{align*}

\section{反常积分}

\subsection*{题目3}
判断积分收敛性:
\[ \int_1^\infty \frac{\ln x}{x^p} dx \quad (p > 0) \]

\subsubsection*{分析}
\begin{table}[h]
\centering
\begin{tabular}{ll}
\toprule
$p$的范围 & 收敛性 \\
\midrule
$p > 1$ & 收敛 \\
$0 < p \leq 1$ & 发散 \\
\bottomrule
\end{tabular}
\end{table}

\subsubsection*{证明}
\begin{itemize}
\item 当$ p = 1 $时:
\[ \int \frac{\ln x}{x} dx = \frac{1}{2}{\ln x}^2 \to \infty \quad (x \to \infty) \]

\item 当$ p \neq 1 $时:
\begin{align*}
\int \frac{\ln x}{x^p} dx &= \frac{x^{1-p}}{1-p} \ln x - \frac{x^{1-p}}{{1-p}^2} + C \\
\text{当} \quad x \to \infty \quad &\text{仅当} \quad p > 1 \quad \text{时收敛}
\end{align*}
\end{itemize}

\section*{附录:常用积分公式}
\begin{align*}
&\int \tan x \, dx = -\ln|\cos x| + C \\
&\int \sec x \, dx = \ln|\sec x + \tan x| + C \\
&\int \frac{dx}{\sqrt{a^2-x^2}} = \arcsin\left(\frac{x}{a}\right) + C \quad (a > 0)
\end{align*}

\end{document}